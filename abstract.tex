\begin{abstract}
\setcounter{page}{2}

Scalability is a term whose beginnings may be traced back to the birth of the trend of using computers in venture and which gradually wins respect in the computer-science environment. Common usage of this term often differs from the proper definition of scalability, leading to misunderstanding or using it exchangeably with other notions carrying a wider scope. The proper understanding together with relevant knowledge can lead to system designs adapted to large scale usage that may become respectively expanded. The study explores the problem of scalability by using the Python programming language with combination of scalable infrastructure of Google. Utilizing Google App Engine and Mercurial, a popular version control system, allowed to probe the common problems and create a networking notes taking application with synchronization feature. The benefit of this work is a case study of the mentioned application, facing the problematics both of design and implementation of scalable application resulting in list of three design practices that can be valuable in a process of creating and scaling computer systems.        

\end{abstract}

