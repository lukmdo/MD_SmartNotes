\chapter{Conclusion}
Writing scalable applications is writing with credit of effort that will pay back when the applications grow. The price may be high and it is important to keep a balance between efforts of making the application scalable with the progress of development. Used by author  the Google App Engine has provided a wonderful sandbox for experiments and eventually development of application called SmartNotes. Taking advantage of production used infrastructure of Google made collected experience a truly practical value. However even Google was building infrastructure elements exposed to wide usage by Google App Engine  with harmony of gradual growth.

Conclusions will be presented by summarizing the completion of aims set in Chapter~\ref{sec:Introduction} each at a time:
\begin{itemize}
\item{\textbf{Clarify the scalability term.} Section~\ref{sec:scalability} pointed the main problems regarding understanding the term of scalability and presented accessible definition.}
\item{\textbf{Write a Python application with scalability in mind.} The base idea of application has been shaped by research done for similar projects and and available tools that could be utilized for implementation. This were presented in Chapter~\ref{sec:related_work}. Sections~\ref{subsec:scalability_on_gae} and~\ref{sec:hg_on_gae} were uncovering some of SmartNotes implementation details with special attention payed to scalability. Then the Application become presented in Chapter~\ref{chap:eval}. The application reached a functional state but still further development will be continued.}
\item{\textbf{Form few best practices for writing scalable applications.}. Using the experience of professional Google App Engine developers \cite{gae_best_practices, mike_malone_quote} the author can state fallowing points that have helped to reach the SmartNotes scalability guidelines.
\begin{itemize}
\item{Cashing. Cashing can has double positive effect on the system once improving the system performance by shorting the response times twice by allowing to reduce the number of reads and writes. Therefore it is a way to reduce data contention generated by high rate writes to the system.}
\item{Data sharding. It shows out that it is a frequent case when non entire data is needed to the users. The application should become designed that way that allows to fit best those needs. That rule also regards writes allowing to bypass the bottleneck of writing to single point.}
\item{Minimizing excess code overhead. Especially performance critical system elements should use well suited tools. When flexibility is generally desirable the fundaments should be rather more solid realizing only destined tasks.}
\item{Entire system concept. Even best designed subsystems wont be much useful and wont deliver required parameters if they cant cooperate well together. A complete system vision is absolutely needed.  }
\end{itemize}
}
\end{itemize} 