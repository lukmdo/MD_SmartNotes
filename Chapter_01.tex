\chapter{Introduction}
\label{sec:Introduction}
Creating network applications nowadays might be a complicated process of design which involves long series of research and experiments or, on the other hand, it might be a fairly simple procedure that takes only a tenth of its time and effort. In both cases, the most frequent evaluation metric is the scalability of the application, more than a thousand of lines of code or the complexity of the model. With the rise in popularity, the usage of network applications increases, which in consequence results in an increase in the frequency that the application is requested to serve its functionality. This growth can be calculated and included at the planning level, thus becoming a good programming practice that places itself close to other widely approved design patterns. With no doubt is scalability starting to play a more important role, being often set at the same level with such issues like portability or security[quote]. The present thesis aims to introduce an application that by its original functionality would heave the potential to become a heavy traffic network application with numerous active users. In this chapter the reader can find an overview of popular notes taking applications, understand the basis of Version Control Systems, and learn about more technical sections regarding the Python programming language as well as terms of scalability in a Google App Engine product.