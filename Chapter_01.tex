\chapter{Introduction}
\label{sec:Introduction}
Creating network applications nowadays might be a complicated process of design which involves long series of research and experiments or, on the other hand, it might be a fairly simple procedure that takes only a tenth of its time and effort. In both cases, a frequent evaluation metric is the scalability of the application, more than a thousand of lines of code or the complexity of the model. With the rise in popularity, the usage of network applications increases, which in consequence results in an increase in frequency that the application is requested to serve its functionality. This growth can be calculated and included at the planing level, thus becoming a good programming practice that places itself close to other widely approved design patterns. With no doubt is scalability starting to play a more important role, being often set at the same level with such issues like portability or security[quote]. The present thesis aims to introduce an application that by its original functionality would heave the potential to become a heavy traffic network application with numerous active users.

\section{Popular notes taking applications}\label{sec:popular_apps} 
blablabla
\subsection{Appx}\label{subsec:x} 
\subsection{Appy}
\subsection{Comparison}
\section{Popular Version Control Systems}\label{sec:popular_vcs}
Although currently there exists a wide variety of VCS (Version Control Systems) offering diverse functionality the true golden age for VCS was started relatively not long ago. It was 2001 when after great success of CVS (Concurrent Version Control) Jim Blandy and Karl Fogel, started a new project aiming to replace CVS by solving already well known inconveniences of CVS, better architecture and cleaner code\cite{hg_book}. This project has been called Subversion also known as SVN by it's commandlne utility name. Although its CVS which probably holds a title of the world's most widely used VCS it is SVN which birth effected in dozens of new and original concepts. Some of this started some time after SVN like Mercurial, Bazzar, Darts other like Git or Guilt needed some time to evolute and gain new developers to join the group of wold famous VCS systems. Nevertheless the first VCS are much older and one the very first ones called SCCS (Source Code Control System) was grounded in 1970s at Bell Labs. At that time computer popularity was not that strong and access to computers was much more restricted what explains the dynamics in which the VCS systems were developed during those days. 

\subsection{Mercurial}\label{subsec:hg}
\subsection{Comparison with other systems}\label{subsec:dvcs_compare}
Alt
\section{Role of programming languge}\label{sec:language}
\subsection{Python}\label{subsec:py}
\subsection{Comparison with other languages}\label{subsec:lang_compare}
\section{Scalability}\label{sec:scalability}
\section{Google App Engine platform}\label{sec:gae_general}