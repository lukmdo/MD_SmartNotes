\chapter{Introduction}
\label{sec:Introduction}
Creating network applications nowadays might be a complicated process of design which involves long series of research and experiments or, on the other hand, it might be a fairly simple procedure that takes only a tenth of its time and effort. In both cases, a frequent evaluation metric is the scalability of the application, more than a thousand of lines of code or the complexity of the model. With the rise in popularity, the usage of network applications increases, which in consequence results in an increase in frequency that the application is requested to serve its functionality. This growth can be calculated and included at the planing level, thus becoming a good programming practice that places itself close to other widely approved design patterns. With no doubt is scalability starting to play a more important role, being often set at the same level with such issues like portability or security[quote]. The present thesis aims to introduce an application that by its original functionality would heave the potential to become a heavy traffic network application with numerous active users.

\section{Popular notes taking applications}\label{sec:popular_apps} 
No matter if we work on some small task plan a holiday or just want to make a list of ideas of our friends birthday and share it secretly with other organizers definitely a computer application might me a useful help.     
Currently users have to rich variety of notes taking application to choose in between. One group of applications can be represent the idea of simple user interface which mimics the well known sticky notes or cork board where the notes are not too long and easy to find. For that kind of applications a good representatives might be the Sticky Notes\footnote{This note taking application was used to be the default one for notes taking under GNOME one of the Linux highly popular desktop environments.}, Knotes\footnote{This application is a component for application called Kontact which is frequently used with KDE Linux desktop environment.} or the Stickies\footnote{A small and handy application which comes with Apple's Mac OS X that has rivals like SketchBox aiming in the customization possibilities.}.They have to offer simple text highlighting, syntax correction and general layout customization. The remaining group of applications has more features to offer like rich text formating with various fonts support and possibility of embedding multimedia elements and hyperlinks. Some of them make use of the Internet accounts where notes might be edited and tagged. In consequence of great diversity of this applications I decided to  present two of them that in my opinion deserve for special attention.

\subsection{Google Notebook}\label{subsec:google_notebook}
This application is one of the Google company products\footnote{Google offer a lots of products. Right next to the browser, email service such applications like Google Calendar, Google Docs, Picassa Web or YouTube reach more and more users.} that is open to the end user without charge or special restrictions. It makes use of typical Google design patterns so its usage is naturally easy for those who are familiar with any other of Google applications. It has some standard elements like bookmarks and tags what shortens the search time when the notes are kept in groups. Apart from that the interface has extended WYSIWYG\footnote{WYSIWYG is an acronym for What You See Is What You Get. It is used when referring to editors that have user friendly interface that operates on some markup language (i.e.. HTML or \TeX) but lets the user manipulate just on how the output will look like and does not requirer knowledge on that markup language.} functionality which make the rich text formating easy for every user.

For various reasons the users might wont to export their notebooks. With Google Notebook is truly easy -- there are tree possibilities: exporting to HTML format, printing or exporting to Google Doc format. What might be considered as even more practical is the opportunity of collaborating on a notebook with other people. That is set up by deciding and marking the notebook as shared and then inviting others to work on it by passing a list of emails. That allows for a great user experience where the users may share their ideas by working on one notebook simultaneously or in separate. That feature needed from Google to implement some simple version control system that would merge the parts into one pice and that would handle the conflict situations. A conflict might happen where two ore more users are editing a same content i.e. same sentence but in different ways that makes further merging operation impossible. The way it is handled by google is presented on picture \dots .The user then has to decide which version is correct one. That is the second of two possibilities that VCS have to offer in that case. The first one is to make prioritizing by the user or detailed modification time and in effect overwriting the conflict situation with the version of the highest priority. That might appear as interesting solution but is not a good practice as it would lead to mistreating others person work without opportunity to notice and react to that situation.

Finally what may seam to be not that important at first glance but in the light of the fact that internet applications got new dynamically growing market of mobile devices Google Notebook can be quite  confident of its position. As a matter of fact that Google already in May 2006 has released the inital version of Google Web Toolkit GWT which allowed for huge gains of Java script and CSS compression caring for cross browser compatibility nowadays GWT is a very powerful tool that lets the designer make the Web based interface and it will work just fine on majority of browsers (including mobile devices) available on the market. That may seam even more important as the idea of having own notebook on mobile devices has a great potential. 

\subsection{Evernote}\label{subsec:evernote}
Evernote is the name of the commercial note taking applications a commercial product that belongs to Evernote Corporation. It has a functional interface, bunch of unique features and still growing group of users. It supports multiple mobile devices just like various operating systems. It allows in a fast manner add, modify and group the notes taken by its users. It makes an interesting approach on how the notes are done. Evernote supports embedding various multimedia elements  to the notes what makes note taking process similar to editing a blog entry. More one of key features of Evernote is that it allows to clip everything that its users find on Internet that pays their attention. It could be some books, cooking recipes or interesting article that they don't have the time to read trough. Later if they want come back to it they can use Evernote on that or some other device with Evernote installed. That feature is especially useful to people that deal with massive  of amount of information that cant be just read or memorized and thanks Evernote  may be stored in a easily searchable way. Despite of that images added to Evernote are also searchable what means that
text on that pictures is recognized as any other text. For example a bussines conntact card might be added to Evernote in couple of seconds by making a photo of it and and later on can be easily found by typing some keywords from that card. That advantage seams to be enormous as by passing a photo of a ticket, bill or any other element with text element  Evernote users got a easy to use jet elastic tool for importing data to their notes that otherwise would become lost or that would need significantly more time to get the data in on the traditional way.

As mentioned at the beginning of this section Evernote is a commercial product what means that the code of it is proprietary software of Evernote Corporation. It may be used without of charge with some limitations like no collaboration option, limited file synchronization formats and lower monthly rate of multimedia elements that may be clipped to the Evernote application. Despite that free users have to accept that they will receive promotion and advertisements  materials in order to use the application. The cost of premium account with some extended options and lower limitations is 5\$ a month or 45\$ for a year fee. That still remains a not too high price on the marked and without doubt Evernote is an interesting tool.      
 
\subsection{Comparison}
Described before programs represent some features that have great potential but defiantly they do not exploit all the interesting features of applications available on the market. Here dynamics is an important factor which tends to fallow the way people use Internet in they daily live. This should be especially taken into account when designing any kind of network application that target to great audience of users. This applications have to keep the speed of the market changes and evolute with its users. That wont guarantee a success but disrespecting that rule might cost the highest cost for a application -- loosing all its users. Other rule that seems to matter a lot is the first user experience which most common is build on the perception of application User Interface. That may decide which application will the user decide to use for longer and which he will never open again. Because currently the user has a wide choice it is or the unique features or the interface which pay highest importance when he will have to pick an application. Because people are naturally tend to have different habits and expectations a rule of keeping things as simple as possible far out perform applications with lots of advanced features and complicated interface. This stays in common for applications that get on popularity. No matter if is a commercial product same rules apply: well designed set of features and simple user interface that can be used straight forward characterize the top rated note taking applications. 
 
\section{Popular Version Control Systems}\label{sec:popular_vcs}
Imagine a group of developers working together on some pice of software.  Most probably they divided they work into functional pieces and made necessary planing. Then they start to code accordingly to the company's coding standards, used methodology or favorite schema. Doubtless they will need to interact not only by exchanging ideas but also by working on same parts of code simultaneously.  At that point they want to work uninterrupted on their code by the same time letting other people to view their work progress or allow to make any modifications that they might want to. That basic need was the primary reason for inventing external software which additionally could tell the to the developer what has changed since the last time he worked on the code or merge the the work of several developers. 

Although currently there exists a wide variety of VCS (Version Control Systems) offering diverse functionality the true golden age for VCS was started relatively not long ago. It was 2001 when after great success of CVS (Concurrent Version Control) Jim Blandy and Karl Fogel, started a new project aiming to replace CVS by solving already well known inconveniences of CVS, better architecture and cleaner code~\cite[page 11]{hg_book}. This project has been called Subversion also known as SVN by it's commandlne utility name. Although its CVS which probably holds a title of the world's most widely used VCS it is SVN which birth effected in dozens of new and original concepts. Some of this started some time after SVN like Mercurial, Bazzar, Darts other like Git or Guilt needed time to evolute and gain new developers to join the group of wold famous VCS systems. Nevertheless the first VCS are much older and one of the very first ones called SCCS (Source Code Control System) was grounded in 1970s at Bell Labs. When taken into account that at that time computer popularity was not that strong and access to computers was much more the dynamics in which the VCS systems were developed during those days wont seam strange. 
\subsection{Mercurial}\label{subsec:hg}
Mercurial is a DVCS (Distributed Version Control System) which is one of VCS models. Describing DVCS in general makes it easier to understand how Mercurial system might be used and to which use cases it suits well. In comparison to SVN or any other centralized VCS where only one main machine contains the repository with its history DVCS make every user to have the those on his hard disk. That in consequence makes each have the write access to the repository with all the available tools. Moreover it allows the user work undisturbed disregarding the server status as he servers the role of server for himself. This simple modification to the concept of centralized VCS was exactly that what dynamically expanding Open Source projects were seeking. Each developer might work on his repository allowing others to use his work by performing merge operation and using work of others just in the same order. No longer network connection to main server was needed in order to have a functional environment. Developers could analyze, create, modify code whenever they wonted to. That was not only a step forward in making VCS more user friendly but also improved the performance of this systems. All metadata concerning repository was placed on users hard disks. There was no reason to connect to the main sever to for example find the date when somebody made the modifications of a class that unexpected stopped to work correctly. This also made this systems more scalable as for the DVCS the central machine, if present, is only used as public main stream version of a project consuming minimum of the machine CPU and disk space. It only role is to allow users to download the most recent version of repository. By having the project with it history saved on every developer hard disk problem of backups was automatically solved and made it perfectly crush secure. On the other hand it might seem that maintaining repository with their history makes the cumulative size of repository much larger. Metadata with the development history is naturally some additional overhead that has to be stored on users disks but typically it does not overlap the three size of the metadata free repository. That is the price that the DVCS users have to agree to pay for the additional functionality which the  system offers. It is also worth to note that this systems by design were encouraging its users to experimenting. Some operations like branching or forking were created just for that purpose. Making frequent tries beyond the main version was a desirable feature in order to keep the history clean on the one hand  and allowing developers to perform as may tries as they might wont to on the other. In that way this systems played important role for  Open Source communities by helping to build and keep the community healthy~\cite{git_talk,svn_talk}. Despite of that aspects DVCS has a bunch of advantages also for commercial projects. Bryan O'Sullivan points in~\cite[page 6]{hg_book} few of them: 
\begin{itemize}
\item{Better availability and reliability for teams that are scattered across the globe.}
\item{Better scalability and easier to maintain.}
\item{Greater flexibility what might be a value for target customer.}
\end{itemize}
What also counts Mercurial is easier to learn and uses similar commands which SVN or CSV were using what eases the transition. Next advantage of Mercurial that is frequently outlined is the very good Microsoft Windows operating system support just like for the other platforms like Linux or Apple OS X. Finally Marcurial has a very efficient HTTP protocol support~\cite{google_hg_git_compare} both as client and server application. This feature counts much if the work on particular repository is tend to be very active in a group of the developers.

\subsection{Mercurial comparison with other systems}\label{subsec:dvcs_compare}
There exist lots of other interesting distributed version control systems which offer interesting functionality. Some of them like atomic commits, GUI\footnote{Graphical User Interface -- it eases the usage of system to users which are not used to work with command line tools.} tools, commits tagging, tracking merge operations in the history or allowing for user defined actions just before or after certain actions performed on repository are only few of which are commonly used for DVCS comparisons~\cite{wiki_dvcs_compare}. With time the already long list~\cite{wiki_dvcs_list} will probably continue to expand. Nevertheless currently Mercurial most popular alternatives are Git and Bazzar. Git comes from Linux kernel developers community and Bazzar is close with GNU\footnote{The name “GNU” is a recursive acronym for “GNU's Not Unix!”. The GNU Project was launched in 1984 to develop a complete Unix-like operating system which is free software and currently is recognizable for GNU/Linux operating system.} developers group. Both are great systems using some slightly different concepts. Both use C language what should have an positive effect on their performance. However it needs compilation and might not be suitable to systems which don't allow for running binary programs. The biggest difference is the level of complexity. On one hand Git provides numerous commands and arguments allowing for full control over repository and it history and on the other Bazzar which motto says "Bazaar adapts to the way you want to work". In that subject Bazzar stays close to Mercurial which allows to use the version control system spending minimum time on learning the proper usage. In the maintenance of a repository Mercurial is absolutely number one. It does not require any additional operations like running \texttt{git-gc}\footnote{bla bla bla} in case of Git and typically uses less disk space than Bazzar needs for the same repository. Apparently all tree systems reached a mature level what can the conclusion after using advanced commands, user interface tools availability and  advanced features as a comparison criteria. The most important reason for making a selection between many of VCS for a developer is frequently offered functionality and the habit of the group where he used to work. That was not the criteria that I followed by choosing a VCS for my application. I needed a lightweight yet powerful python based system with good support for running over HTTP requests. That filter was strict enough to choose Mercurial over other available systems.     

\section{Role of programming languge}\label{sec:language}
\subsection{Python}\label{subsec:py}
\subsection{Comparison with other languages}\label{subsec:lang_compare}
\section{Scalability}\label{sec:scalability}
To define the scalability term I will base on work done by Cal Henderson in his book~\cite[pages 203--204]{build_scalable} where he points thee characteristics that scalable system should have:
\begin{itemize}
\item{The system can accommodate increased usage.}
\item{The system can accommodate increased dataset.}
\item{The system is maintainable.}
\end{itemize}
That definition does not require much additional explanations. However it is a basic criteria that should help to understand the term meaning but does not exploit the problematics standing behind it. It may be crucial to understand that scalability as a process of planning should be grounded before the increase will happen. In plain words scalability pattern could stand for 'increase ready' what in the real-world project has some certain consequences:
\begin{itemize}
\item{Capital investment will be made.}
\item{The system will become more complex.}
\item{Maintenance costs will increase.}
\item{Time will be required to act.}
\end{itemize}
Accordingly to the author of  "Scalable Internet Architectures"~\cite[page 8]{scalable_arch} above four points are guaranteed when scaling any system. Therefore scalability has it own cost that should make the desired growth on scale more profitable if not only possible. As horizontal scaling means expenses for extra number of elements that brand or model we have already in use. Vertical   





\section{Google App Engine platform}\label{sec:gae_general}