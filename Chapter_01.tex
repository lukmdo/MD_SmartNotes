\chapter{Introduction}
\label{sec:Introduction}
Creating network applications nowadays might be a complicated process of design which involves long series of research and experiments or, on the other hand, it might be a fairly simple procedure that takes only a tenth of its time and effort. In both cases, the most frequent evaluation metric is the scalability of the application, more than a thousand of lines of code or the complexity of the model. With the rise in popularity, the usage of network applications increases, which in consequence results in an increase in the frequency that the application is requested to serve its functionality. This growth can be calculated and included at the planning level, thus becoming a good programming practice that places itself close to other widely approved design patterns. With no doubt is scalability starting to play a more important role, being often set at the same level with such issues like portability or security\cite{scalable_delft}. The present thesis aims to introduce an application that by its original functionality would heave the potential to become a heavy traffic network application with numerous active users. 

The idea of SmartNotes, how I have called the application, has been inspired by the Google Notebook application which was being actively developed until January 2009, when Google announced that further development work on this project is stopped. Google Notebook has an interesting interface and features, some of them already motioned in section~\ref{subsec:google_notebook}. What was still missing from the functionality until January 2009 was comfortable notes usage without network connectivity. The aim of the subject of the thesis was to use the idea of making a truly scalable notes taking application that would be even more flexible by offering offline functionality and synchronization feature. All of this delivered in elegant user interface that pays special attention to easiness and functionality.

SmartNotes and the iSmartNotes client are mend to be Open Source projects that after my graduation could become developed by people who support it idea, willing to contribute and see their ideas working for the other users. 
To make that possible I used one of the Version Control Systems called git and have released the code under the LGPL license. The code is available on Github one of the most poplar code hosting services which is widely used in the Open Source community. For easy visualization what is going in the project, marking important changes and gathering links and information regarding the project and care for users opinion I decided to use some additional services. For that purpose SmartNotes have its own:
\begin{itemize}
\item{\textbf{Project home page}. At \url{http://smart-notes.appspot.com/} users can find a brief overview what the project does and how they can use it.}
\item{\textbf{SmartNotes development blog}. Project home page makes it easy for the users interested in source code to find the blog address \url{http://blog-smart-notes.blogspot.com/} which aims to to provide an official area where some important design decision can be taken, developers can exchange their opinions and share their experience.}
\item{\textbf{Ohloh SmartNotes profile}. It is linked from the project homepage and can be found at \url{http://www.ohloh.net/p/SmartNotes}. This service offers a friendly user interface which helps users of  diverse level of technical knowledge to stay updated with the project status.}
\item{\textbf{Get Satsfaction SmartNotes profile}. The project home page provides a feedback form where the users can ask questions, share ideas, report problems or give praise to the application  that later on can be accessed and responded at \url{http://getsatisfaction.com/smartnotes/}.}
\end{itemize}
This all are users or developers oriented services that should ease the process of finding informations regarding SmartNotes application and provide a official communication channel. This way it is far easier to react for the user signals which plays an important part of building a strong group of satisfied users with the potential to increase. Building a reliable service that its users relay on is one of the key elements allowing to explore the scalability of the application architecture.
 
